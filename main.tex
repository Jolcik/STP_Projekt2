\documentclass{article}
\usepackage[utf8]{inputenc}

\usepackage{polski}
\usepackage{amsmath}
\usepackage{graphicx}
\usepackage{mathtools}

\title{STP projekt nr 2}
\author{Piotr Kostrzeński}
\date{nr indeksu 300421}

\begin{document}

\maketitle

\section*{Wstęp}
Zadany obiekt regulacji jest opisany transmitancją:
\[
    G(s) = \frac{
        K_0 e^{-T_0s}
    }{
        (T_1s+1)(T_2s+1)
    }
\]
gdzie $K_0 = 3,4, T_0 = 5, T_1 = 1,75, T_2 = 5,51$/

\section*{Zadanie 1}
W ramach tego zadania wyznaczę transmitancję dyskretną $G(z)$ na podstawie zadanej transmitancji ciągłej $G(s)$. Zastosuję przy tym ekstrapolator zerowego rzędu oraz okres próbkowania $T_p = 0,5s$. Zadaną transmitancję ciągłą można także przedstawić jako:
\[
    G(s) = e^{-T_0s} \cdot G_0(s)
\]
gdzie:
\[
    G_0(s) = \frac{K_0}{(T_1s+1)(T_2s+1)}
\]
Transmitancję $G_0(s)$ rozkładam na ułamki proste:
\begin{align*}
    G_0(s) & = \frac{K_0}{(T_1s+1)(T_2s+1)} = \frac{\frac{K_0}{T_1T_2}}{(s+\frac{1}{T_1})(s+\frac{1}{T_2})} =  \frac{A}{s+\frac{1}{T_1}} + \frac{B}{s+\frac{1}{T_2}} = \\
    & = \frac{A(s+\frac{1}{T_2}) + B(s+\frac{1}{T_1})}{(s+\frac{1}{T_1})(s+\frac{1}{T_2})} = \frac{(A+B)s + \frac{A}{T_2} + \frac{B}{T_1}}{(s+\frac{1}{T_1})(s+\frac{1}{T_2})}
\end{align*}

\noindent
Przyrównując współczynniki wielomianu w liczniku ułamka otrzymuję układ równań:
\[
    \begin{cases}
        A + B = 0 \\
        \frac{A}{T_2} + \frac{B}{T_1} = \frac{K_0}{T_1T_2}
    \end{cases}
\]
\[
    \begin{cases}
        A + B = 0 \\
        \frac{AT_1 + BT_2}{T_1T_2} = \frac{K_0}{T_1T_2}
    \end{cases}
\]
\[
    \begin{cases}
        A + B = 0 \implies B = - A \\
        AT_1 + BT_2 = K_0
    \end{cases}
\]
\[ AT_1 - AT_2 = K_0\]
\[ (T_1 - T_2)A = K_0\]
\[ A = \frac{K_0}{T_1 - T_2} \]
\[ B = -A = -\frac{K_0}{T_1 - T_2}\]
Podsumowując współczynniki:
\[
    \begin{cases}
         A = \frac{K_0}{T_1 - T_2} \\
         B = -\frac{K_0}{T_1 - T_2}
    \end{cases}
\]
co mi daje:
\[
    G_0(s) = \frac{\frac{K_0}{T_1 - T_2}}{s+\frac{1}{T_1}} + \frac{-\frac{K_0}{T_1 - T_2}}{s+\frac{1}{T_2}}
\]
Transmitancję ciągłą zamieniam na dyskretną według wzoru:
\[
    G_0(z) = \frac{z-1}{z} \cdot Z \left( \frac{G_0(s)}{s} \right)
\]
Daje to:

\begin{align*}
    G_0(z) & = \frac{z-1}{z} \cdot Z \left( \frac{\frac{K_0}{T_1 - T_2}}{s(s+\frac{1}{T_1})} + \frac{-\frac{K_0}{T_1 - T_2}}{s(s+\frac{1}{T_2})} \right) = \\
    & = \frac{z-1}{z} \cdot Z \left( \frac{T_1K_0}{T_1 - T_2}\frac{\frac{1}{T_1}}{s(s+\frac{1}{T_1})} - \frac{T_2K_0}{T_1 - T_2}\frac{\frac{1}{T_2}}{s(s+\frac{1}{T_2})} \right) = \\
    & = \frac{z-1}{z} \left[ 
        \frac{T_1K_0}{T_1 - T_2}\frac{z(1-e^{-\frac{1}{2T_1}})}{(z-1)(z-e^{-\frac{1}{2T_1}})} -
        \frac{T_2K_0}{T_1 - T_2}\frac{z(1-e^{-\frac{1}{2T_2}})}{(z-1)(z-e^{-\frac{1}{2T_2}})}
    \right] \\
    & = \frac{T_1K_0}{T_1 - T_2} \cdot \frac{1-e^{-\frac{1}{2T_1}}}{(z-e^{-\frac{1}{2T_1}})} - \frac{T_2K_0}{T_1 - T_2} \cdot \frac{1-e^{-\frac{1}{2T_2}}}{(z-e^{-\frac{1}{2T_2}})} \\
    & = \frac{K_0}{T_1 - T_2} \left( \frac{T_1(1-e^{-\frac{1}{2T_1}})}{(z-e^{-\frac{1}{2T_1}})} - \frac{T_2(1-e^{-\frac{1}{2T_2}})}{(z-e^{-\frac{1}{2T_2}})} \right) \\ 
    & = \frac{K_0}{T_1 - T_2} \cdot
    \frac{
        T_1(1-e^{-\frac{1}{2T_1}})(z-e^{-\frac{1}{2T_2}}) -
        T_2(1-e^{-\frac{1}{2T_2}})(z-e^{-\frac{1}{2T_1}})
    }{
        (z-e^{-\frac{1}{2T_1}})(z-e^{-\frac{1}{2T_2}})
    } \\
    & = \frac{K_0}{T_1 - T_2} \cdot
    \frac{
        \left[T_1(1-e^{-\frac{1}{2T_1}})-T_2(1-e^{-\frac{1}{2T_2}})\right]z -
        e^{-\frac{1}{2T_1}}T_2(1-e^{-\frac{1}{2T_2}}) - e^{-\frac{1}{2T_2}}T_1(1-e^{-\frac{1}{2T_1}})
    }{
        (z-e^{-\frac{1}{2T_1}})(z-e^{-\frac{1}{2T_2}})
    } 
\end{align*}
\noindent
Ostatecznie:
\begin{align*}
    G_0(z) = \frac{K_0}{T_1 - T_2} \cdot
    \frac{
        \left[T_1(1-e^{-\frac{1}{2T_1}})-T_2(1-e^{-\frac{1}{2T_2}})\right]z -
        e^{-\frac{1}{2T_1}}T_2(1-e^{-\frac{1}{2T_2}}) - e^{-\frac{1}{2T_2}}T_1(1-e^{-\frac{1}{2T_1}})
    }{
        z^2 - (e^{-\frac{1}{2T_1}} + e^{-\frac{1}{2T_2}})z + e^{-\frac{1}{2T_1}}e^{-\frac{1}{2T_2}}
    }
\end{align*}

\noindent
Po podstawieniu liczb otrzymuję taką transmitancję dyskretną:
\begin{align*}
    G_0(z) & = -0,90425 \cdot \frac{
       (0,43491 - 0,47798)z + 0,35919 - 0,39719 
    }{
       z^2 - 1,66473z + 0,68628
    } \\
    & = -0,90425 \cdot \frac{
       -0,04307z - 0,038
    }{
       z^2 - 1,66473z + 0,68628
    } \\ 
    & = \frac{
       0,03894z + 0,0343 
    }{
       z^2 - 1,66473z + 0,68628
    }
\end{align*}
Otrzymaną transmitancję sprawdziłem także w środowisku MATLAB. Najpierw przekształciłem transmitancję ciągłą:
\begin{align*}
    G_0(s) & = \frac{\frac{K_0}{T_1T_2}}{(s+\frac{1}{T_1})(s+\frac{1}{T_2})} \\
    & = \frac{\frac{K_0}{T_1T_2}}{s^2+(\frac{1}{T_1} + \frac{1}{T_2})s + \frac{1}{T_1T_2}}
\end{align*}
Po podstawieniu liczb:
\[
G_0(s) = \frac{0,3526}{s^2+0,7529s + 0,1037}
\]
Transmitancję w takiej postaci można w środowisku MATLAB zamienić na transmitancję dyskretną w poniższy sposób:
\begin{verbatim}
    H = tf([0.3526], [1 0.7529 0.1037]);
    Hd = c2d(H, 0.5);
\end{verbatim}
Wynik powyższych obliczeń wygląda nastepująco:
\begin{verbatim}
    Hd =
 
       0.03895 z + 0.03435
      ----------------------
      z^2 - 1.665 z + 0.6863
\end{verbatim}
\noindent
Co potwierdza poprawność wyżej wykonanych obliczeń. Pozostało jeszcze zająć się pominiętym póki co opóźnieniem. Po przyjęciu okresu próbkowania $T_p = 0,5s$ opóźnieniu ciągłemu $e^{-5s}$ odpowiada opóźnienie dyskretne $z^{-10}$. Stąd ostatecznie, transmitancja dyskretna $G(z)$ przyjmuje postać:
\[
    G(z) = \frac{
       0,03894z + 0,0343 
    }{
       z^2 - 1,66473z + 0,68628
    } z^{-10}
\]

\clearpage \noindent
Obie transmitancje zostały zaimplementowane w środowisku MATLAB. Nastepnie sprawdzono ich odpowiedzi skokowe. Wyglądają one następująco: \\
\includegraphics[width=\textwidth]{STP_projekt2_zad1.png}

\noindent
Jak można zauważyć, transmitancję się praktycznie ze sobą pokrywają (oczywiście biorąc pod uwagę to że jedna jest dyskretna). Dowodzi to poprawności przeprowadzonej dyskretyzacji transmitancji. Współczynnik wzmocnienia statycznego transmitancji można policzyć z poniższych wzorów, dla transmitancji ciągłej i dyskretnej:
\[
    K_{stat} = \lim_{s \to 0} G(s), K_{stat} = \lim_{z \to 1} G(z)
\]
Stąd, dla transmitancji ciągłej:
\[
    K_{stat} = \lim_{s \to 0}  \frac{0,3526}{s^2+0,7529s + 0,1037} e^{-5s} = \frac{0,3526}{0,1037} \approx 3,4
\]
i dla dyskretnej:
\begin{align*}
    K_{stat} & = \lim_{z \to 1} \frac{
       0,03894z + 0,0343 
    }{
       z^2 - 1,66473z + 0,68628
    } z^{-10} \\
    & = \frac{
       0,03894 + 0,0343 
    }{
       1^2 - 1,66473\cdot1 + 0,68628
    } 1^{-10} = \frac{0,07324}{0.02155} \approx 3,3986 \approx 3,4
\end{align*}

\noindent
Obliczone odpowiedzi skokowe są prawie identyczne (błędy związane z zaokrągleniem). Zatem wartości współczynnika wzmocnienia statycznego transmitancji ciągłej i dyskretnej są sobie równe. Można to także zauważyć, na wykresie z odpowiedziami skokowymi. Widać na nim, że odpowiedź skokowa dla obu transmitancji stabilizuje się po pewnym czasie na obliczonych wartościach.

\section*{Zadanie 2}
W tym zadaniu na podstawie transmitancji dyskretnej wyznaczę równanie różnicowe służące do obliczenia wielkości $y(k)$ na podstawie sygnałów wejściowych i wyjściowych z chwil poprzednich o ogólnym wzorze:
\[
    y(k) = \sum_{i=1}^{n}b_i y(k-i) + \sum_{i=1}^{m}c_i u(k-i)
\]
Transmitancja dyskretna przyjmuje poniższą postać:
\[
    G(z) = \frac{Y(z)}{U(z)} = \frac{
       0,03894z + 0,0343 
    }{
       z^2 - 1,66473z + 0,68628
    } z^{-10}
\]
Równanie różnicowe otrzymuję przekształcając powyższe równanie:
\[
    \frac{Y(z)}{U(z)}  = \frac{
       0,03894z + 0,0343 
    }{
       z^2 - 1,66473z + 0,68628
    } z^{-10}
\]
\[
    Y(z)(z^2 - 1,66473z + 0,68628)  = U(z)(0,03894z + 0,0343)z^{-10}
\]
\[
    Y(z)(z^2 - 1,66473z + 0,68628)  = U(z)(0,03894z^{-9} + 0,0343z^{-10})
\]
W dziedzinie czasu dyskretnego mam zatem:
\[
    y(k+2) - 1,66473y(k+1) + 0,68628y(k) = 0,03894u(k-9) + 0,0343u(k-10)
\]
\[
    y(k+2) = 1,66473y(k+1) - 0,68628y(k) + 0,03894u(k-9) + 0,0343u(k-10)
\]
Stosując podstawienie $k := k-2$ otrzymuję:
\[
    y(k) = 1,66473y(k-1) - 0,68628y(k-2) + 0,03894u(k-11) + 0,0343u(k-12)
\]
Jest to ostateczna forma żądanego równania różnicowego. Pozwala ona na obliczenie wielkości $y(k)$ na podstawie sygnałów wejściowych i wyjściowych z chwil poprzednich.


\clearpage
\section*{Zadanie 3}
W tym zadaniu, dla danego obiektu dobiorę ciągły regulator PID metodą Zieglera-Nicholsa. Metoda Zieglera-Nicholsa polega na tym, że zerujemy człon całkujący i różniczkujący regulatora PID, a następnie zwiększamy współczynnik proporcjonalny do momentu otrzymania wzmocnienia krytycznego $K_k$. Wzmocnienie krytyczne oznacza, że układ oscyluje ze stałym okresem oraz amplitudą. Okres tych oscylacji oznaczam jako $T_k$. Wtedy, według metody otrzymujemy podane wartości wwspółczynników:
\begin{align*}
    K_r = 0,6K_k \\
    T_i = 0,5T_k \\
    T_d = 0,12T_k
\end{align*}
Kod symulujący wygląda następująco:
\begin{verbatim}
    sys_ciagly = tf([0.3526], [1 0.7529 0.1037], 'InputDelay', 5);
    regulator = pid(Kr, Ki, Kd);
    sprzezenie_zwrotne = feedback(regulator*sys_ciagly, 1);
\end{verbatim}
Symulację zaczynam od $K_r = 1$: \\
\includegraphics[width=\textwidth]{STP_projekt2_zad3_1.000.png}

\clearpage \noindent
Obiekt jest wtedy niestabilny, trzeba zatem zmniejszyć wzmocnienie. Zmniejszam jego wartość do $K_r = 0,5$:
\begin{center}
    \includegraphics[width=0.93\textwidth]{STP_projekt2_zad3_0.500.png}
\end{center}

\noindent
Teraz układ się stabilizuje na zbyt niskiej wartości (uchyb ustalony) oraz nie ma oscylacji o stałej amplitudzie, zwiększam zatem $K_r$ do $0,75$: 
\begin{center}
    \includegraphics[width=0.93\textwidth]{STP_projekt2_zad3_0.750.png}
\end{center}

\clearpage \noindent
Za duże wzmocnienie więc je zmniejszam do $K_r = 0,625$:
\begin{center}
    \includegraphics[width=0.93\textwidth]{STP_projekt2_zad3_0.625.png}
\end{center}
Tym razem jest ono za małe, więc je zwiększam do $K_r = 0,650$
\begin{center}
    \includegraphics[width=0.93\textwidth]{STP_projekt2_zad3_0.650.png}
\end{center}

\clearpage \noindent
Dalej za małe, zwiększam do $K_r = 0,675$
\begin{center}
    \includegraphics[width=0.93\textwidth]{STP_projekt2_zad3_0.675.png}
\end{center}
Wzmocnienie $K_r = 0,675$ okazuje się poprawnym. Układ oscyluje ze stałą amplitudą wokół jednej wartości. Oznacza, to że jest to wzmocnienie krytyczne. Okres oscylacji wynosi 20 sekund. Stąd, mam parametry:
\begin{align*}
    K_k = 0,675 \\
    T_k = 20
\end{align*}
Na tej podstawie mogę wyliczyć parametry regulatora:
\begin{align*}
    K_r & = 0,6K_k = 0,6 \cdot 0,675 = 0,405 \\
    T_i & = 0,5T_k = 0,5 \cdot 20 = 10 \\
    T_d & = 0,12T_k = 0,12 \cdot 20 = 2,4
\end{align*}
Mogę także wyznaczyć wartości parametrów dyskretnego regulatora PID definiowanego poprzez wzory:
\begin{align*}
    r_2 & = \frac{KT_d}{T} = \frac{0,405 \cdot 2,4}{0,5} = 1,944 \\
    r_1 & = K \left( \frac{T}{2T_i}-2\frac{T_d}{T}-1 \right) = 0,405 \cdot (0,025 - 9,6 - 1) = -4,2829 \\
    r_0 & = K \left( 1 + \frac{T}{2T_i} + \frac{T_d}{T} \right) = 0,405 \cdot ( 1 + 0,025 + 4,8) = 2,3591
\end{align*}

\clearpage \noindent
Sprawdzam działanie regulatora po zastosowaniu obliczonych parametrów:
\begin{center}
    \includegraphics[width=0.93\textwidth]{STP_projekt2_zad3_0.405.png}
\end{center}
Wyjście obiektu ewidentnie dąży do zadanej wartości. Mimo wszystko, jakość regulacji jest dosyć słaba. Wystarczy jednak obniżyć wartości parametru $T_d$ do $T_d=1,2$, co powoduje otrzymanie o wiele lepszej trajektorii:
\begin{center}
    \includegraphics[width=0.93\textwidth]{STP_projekt2_zad3_0.405_2.png}
\end{center}
Trajektorię można jeszcze poprawić zwiększając czas całkowania do $T_i = 20$:
\begin{center}
    \includegraphics[width=0.93\textwidth]{STP_projekt2_zad3_0.405_3.png}
\end{center}
Dzięki temu osiągnięto dużo mniejsze przeregulowanie przy podobnym czasie regulacji. \\

\noindent
Podsumowując, po dwóch małych poprawkach, za pomocą Metody Zieglera-Nicholsa zostały wyznaczone takie parametry regulatora PID:
\begin{align*}
    K_r & = 0,405 \\
    T_i & = 20 \\
    T_d & = 1,2
\end{align*}
Wtedy też, parametry dyskretnego regulatora będą równe:
\begin{align*}
    r_2 & = \frac{KT_d}{T} = \frac{0,405 \cdot 2,4}{0,5} = 1,944 \\
    r_1 & = K \left( \frac{T}{2T_i}-2\frac{T_d}{T}-1 \right) = 0,405 \cdot (0,0125 - 4,8 - 1) = -2,3439 \\
    r_0 & = K \left( 1 + \frac{T}{2T_i} + \frac{T_d}{T} \right) = 0,405 \cdot ( 1 + 0,0125 + 2,4) = 1,3821
\end{align*}

\clearpage
\section*{Zadanie 4 - regulator PID}
W tej części zadania zaimplementuję w środowisku MATLAB program do symulacji cyfrowego algorytmu PID. \\

\noindent
Zaczynam od zdefiniowania współczynników regulatora PID:
\begin{verbatim}
    T = 0.5;
    Kr = 0.405;
    Ti = 10;
    Td = 2.4;
\end{verbatim}
Oraz według wzorów:
\begin{align*}
    r_2 & = \frac{KT_d}{T} \\
    r_1 & = K \left( \frac{T}{2T_i}-2\frac{T_d}{T}-1 \right) \\
    r_0 & = K \left( 1 + \frac{T}{2T_i} + \frac{T_d}{T} \right) 
\end{align*}
liczę współczynniki dyskretne:
\begin{verbatim}
    r2 = (Kr*Td)/T;
    r1 = Kr*(T/(2*Ti) - 2*(Td/T) - 1);
    r0 = Kr*(1 + T/(2*Ti) + Td/T);
\end{verbatim}
Definiuję też czas trwania symulacji w chwilach dyskretnych $k$:
\begin{verbatim}
    kk = 200;
\end{verbatim}
Następnie na podstawie tej wielkości definiuję wcześniej wektory wejścia, wyjścia oraz uchybu:
\begin{verbatim}
    u = zeros(kk, 1);
    y = zeros(kk, 1);
    e = zeros(kk, 1);
\end{verbatim}
Potem definiuję wartości zadaną wyjścia obiektu. Na początku jest ono równe 0, po to aby w chwili $k=15$ zmienić się skokowo do wartości 1. Chwila $k$ jest równa 15, ponieważ ze względu na równanie różnicowe układ naprawdę zaczyna działać dla $k>12$. Stąd, dla pewności, żeby wszelkie zmienne zdążyły się dobrze zainicjalizować skok ustawiłem w chwili $k=15$:
\begin{verbatim}
    y_zad = zeros(kk, 1);
    y_zad(15:kk) = 1; 
\end{verbatim}

\noindent
Ostatecznie, pozostaje przeprowadzenie samej symulacji. Równanie różnicowe modelu przyjmuje postać:
\[
    y(k) = 1,66473y(k-1) - 0,68628y(k-2) + 0,03894u(k-11) + 0,0343u(k-12)
\]
Ostatni wyraz jest zależny od $u(k-12)$ dlatego pętlę symulacji zaczynam od $k=13$:
\begin{verbatim}
    for k = 13:kk
        ...
    end
\end{verbatim}
Wewnątrz pętli, na początku według podanego równania różnicowego symuluję zachowanie obiektu:
\begin{verbatim}
    y(k) = 1.66473*y(k-1) - 0.68628*y(k-2) 
                                + 0.03894*u(k-11) + 0.0343*u(k-12);
\end{verbatim}
Potem obliczam uchyb regulacji:
\begin{verbatim}
    e(k) = y_zad(k)-y(k);
\end{verbatim}
Na końcu, według równania różnicowego na sterowanie:
\[
    u(k) = r_2e(k-2) + r_1e(k-1) + r_0e(k) + u(k-1)
\]
liczę nową wartość sterowania dla obiektu:
\begin{verbatim}
    u(k) = r2*e(k-2) + r1*e(k-1) + r0*e(k) + u(k-1);
\end{verbatim}
Łącznie, pętla symulacji przedstawia się jak poniżej:
\begin{verbatim}
    for k = 13:kk
        y(k) = 1.66473*y(k-1) - 0.68628*y(k-2) 
                                    + 0.03894*u(k-11) + 0.0343*u(k-12);
        e(k) = y_zad(k) - y(k);
        u(k) = r2*e(k-2) + r1*e(k-1) + r0*e(k) + u(k-1);
    end
\end{verbatim}

\noindent
Na końcu pliku zamieszczony jest także kod generujący wykresy w MATLABie, ale nie służy on symulacji cyfrowego algorytmu PID, dlatego pominę jego treść oraz wyjaśnienie. \\

\noindent
Używając powyższego kodu przeprowadziłem symulację działania regulatora. Zauważyć można na pewno, że model dyskretny ma troszkę inną specyfikę. Dokładniej, po przeprowadzeniu kilku eksperymentów wzmocnienie krytyczne dla regulatora dyskretnego jest równe $K_k = 0,657$ (w przypadku regulatora ciągłego było to $K_k = 0,675$). Opisana wyżej różnica wynika najprawdopodobniej z przyjętych przeze mnie zaokrągleń. Trajektoria w takim przypadku została przedstawiona na kolejnej stronie.

\clearpage
\begin{center}
    \includegraphics[width=0.9\textwidth]{STP_projekt2_zad4_pid_1.png}
\end{center}
Okres oscylacji jest taki sam jak w przypadku ciągłego regulatora. Po wyznaczeniu prametrów według metody Zieglera-Nicholsa (jedyna różnica to $Kr = 0,3402$) trajektoria wygląda jak poniżej:
\begin{center}
    \includegraphics[width=0.9\textwidth]{STP_projekt2_zad4_pid_2.png}
\end{center}
Jak widać, otrzymana jakość regulacji jest akceptowalna, dlatego tak ją zostawiam.


\section*{Zadanie 4 - regulator DMC}
W drugiej części zadania zaimplementuję symulację cyfrowego algorytmu predykcyjnego DMC w wersji analitycznej, bez ograniczeń. \\

\noindent
Jako że w algorytmie  DMC  obiekt jest  modelowany dyskretnymi  odpowiedziami skokowymi, na samym początku należy przeprowadzić symulację odpowiedzi skokowej dyskretnego modelu:
\begin{verbatim}
    sys_dyskretny = tf([0.03894 0.0343], [1 -1.66473 0.68628],
                                        0.5, 'InputDelay', 10);
    s = step(sys_dyskretny, 100);
\end{verbatim}

\noindent
Na tej podstawie otrzymuję wektor $s$. Następnie definiuję wartości parametrów algorytmu:
\begin{verbatim}
    N = 10;
    Nu = 10;
    D = 10;
    lambda = 1;
\end{verbatim}

\noindent
Do realizacji algorytmu DMC potrzebna będzie macierz $M^P$ o wymiarowości $N \times D-1$ i strukturze:
\begin{equation*}
    M^P = \begin{bmatrix}
        s_2 - s_1 & s_3 - s_2 & \dots & s_D - S_{D-1} \\
        s_3 - s_1 & s_4 - s_2 & \dots & s_{D+1} - S_{D-1} \\
        \vdots & \vdots & \ddots & \vdots \\
        s_{N+1} - s_1 & s_{N+2} - s_2 & \dots & s_{N+D+1} - S_{D-1} \\
    \end{bmatrix}
\end{equation*}

\noindent
W MATLABie generuję ją w poniższy sposób:
\begin{verbatim}
    Mp = zeros(N, D-1);
    for i = 1:N
        for j = 1:D-1
            if i + j < D
                Mp(i, j) = s(i + j) - s(j);
            else
                Mp(i, j) = s(D) - s(j);
            end
        end
    end
\end{verbatim}

\noindent
Potrzebna będzie także macierz dynamiczna $M$ o wymiarowości $N \times N_u$ i strukturze:
\begin{equation*}
    M^P = \begin{bmatrix}
        s_1 & 0 & \dots & 0 \\
        s_2 & s_1 & \dots & 0 \\
        \vdots & \vdots & \ddots & \vdots \\
        s_N & s_{N-1} & \dots & s_{N-N_u+1} \\
    \end{bmatrix}
\end{equation*}

\clearpage \noindent
Jej implementacja w środowisku MATLAB przedstawia się poniższo:
\begin{verbatim}
    M = zeros(N, Nu);
    for i = 1:N
        for j = 1:i
            if j <= Nu
                M(i, j) = s(i - j + 1);
            end
        end
    end
\end{verbatim}

\noindent
Idąc dalej, do realizacji algorytmu będzie macierz $K$:
\[
    K = (M^T \Psi M + \Lambda)^{-1} M^T \Psi
\]
gdzie $\Lambda$ to macierz jednostkowa pomnożona przez parametr $\lambda$. Obliczenie macierzy w MATLABie:
\begin{verbatim}
    I = eye(Nu);
    K = (((M')*M + lambda*I)^(-1))*(M');
\end{verbatim}

\noindent
Do obliczenia sterowania według prawa regulacji:
\[
    \Delta u(k) = k^e (y^{zad}(k) - y(k)) - \sum_{j=1}^{D-1}k_j^u \Delta u(k-j)
\]
gdzie
\[
    k_j^u = \overline{K}_1 M_j^P
\]
\[
    k^e = \sum_{p=1}^N k_{1, p}
\]
\noindent
potrzebne będą wektory $k^e$ oraz $k^u$. Obliczam je jak poniżej:
\begin{verbatim}
    ku = K(1, :)*Mp;
    ke = sum(K(1, :));
\end{verbatim}

\noindent
Dalsza część kodu jest podobna do regulatora PID, na początku inicjalizuję zmienne:

\begin{verbatim}
    kk = 120;
    u = zeros(kk, 1);
    d_upk = zeros(1, D-1);
    y = zeros(kk, 1);
    y_zad = zeros(kk, 1);
    y_zad(15:kk) = 1; 
    e = zeros(kk, 1);
\end{verbatim}

\clearpage \noindent
Następnie piszę główną pętlę symulacji (oprócz obliczania sygnału sterującego):
\begin{verbatim}
    for k = 13:kk
        y(k) = 1.66473*y(k-1) - 0.68628*y(k-2) 
                                    + 0.03894*u(k-11) + 0.0343*u(k-12);
        e(k) = y_zad(k)-y(k);
        
        ...
    end
\end{verbatim}

\noindent
W przypadku regulatora DMC obliczanie wartości sygnału sterującego $u$ wygląda tak jak na przytoczonym wcześniej prawie regulacji. Na początku przesuwam wszystkie poprzednie przyrosty sterowania $\Delta u$ w wektorze przechowującym te wartości:
\begin{verbatim}
    d_upk(2:D-1) = d_upk(1:D-2);
\end{verbatim}

\noindent
Następnie obliczam przyrost sygnału sterującego $\Delta u$ oraz zapisuję go do wektora przechowującego te przyrosty:
\begin{verbatim}
    d_uk = ke*e(k) - ku*d_upk';
    d_upk(1) = d_uk;
\end{verbatim}

\noindent
Na końcu obliczam nową wartość sygnału sterującgo $u$:
\begin{verbatim}
    u(k) = u(k-1) + d_uk;
\end{verbatim}

\noindent
Podsumowując, petla symulacji przedstawia się jak poniżej:
\begin{verbatim}
    for k = 13:kk
        y(k) = 1.66473*y(k-1) - 0.68628*y(k-2) 
                                    + 0.03894*u(k-11) + 0.0343*u(k-12);
        e(k) = y_zad(k)-y(k);
        
        d_upk(2:D-1) = d_upk(1:D-2);
        d_uk = ke*e(k) - ku*d_upk';
        d_upk(1) = d_uk;
        u(k) = u(k-1) + d_uk;
    end
\end{verbatim}

\noindent
Reszta kodu w pliku zajmuje się wygenerowaniem oraz zapisaniem wykresu obrazującym działanie regulatora. Nie ma to związku z samym algorytmem dlatego pomijam ten fragment. \\

\noindent
W ten sposób napisałem program pozwalający zasymulować działanie algorytmu DMC w wersji analitycznej, bez ograniczeń. W następnym zadaniu użyję go przy poszukiwaniu odpowiednich parametrów regulatora.



\section*{Zadanie 5}
W ramach tego zadania dobiorę parametry algorytmu DMC na podstawie przeprowadzonych odpowiednio symulacji.

\section*{a)}
Na początku określę horyzont dynamiki $D$. Wyznaczanie zaczynam dla $\lambda = 1$ oraz założenia, że długości horyzontów predykcji i sterowania są takie same jak horyzont dynamiki ($N_u = N = D$). \\

\noindent
Sprawdzam po jakiej ilości chwil dyskretnych odpowiedzi skokowej obiektu wartość wyjścia się stabilizuje. Na tej podstawie ustalę wartość horyzontu. Odpowiedź skokowa wygląda jak poniżej:
\begin{center}
    \includegraphics[width=\textwidth]{STP_projekt2_zad1.png}
\end{center}
Jak można zauważyć na wykresie, wartość odpowiedzi skokowej stabilizuj się mniej po 40 sekundach, co odpowiada 80 chwilom dyskretnym (ponieważ $T_p = 0,5s$). Sprawdzam więc jak przedstawia się wyjście obiektu dla takiej wartości horyzontu dynamiki (wykres na następnej stronie).

\clearpage \noindent
\begin{center}
    \includegraphics[width=0.96\textwidth]{STP_projekt2_zad5_a_Dn80.png}
\end{center}

\noindent
Dla pewności sprawdzę także inne wartości horyzontu dynamiki. Na początku większa wartość $D = 90$:
\begin{center}
    \includegraphics[width=0.96\textwidth]{STP_projekt2_zad5_a_Dn90.png}
\end{center}

\noindent
Wyjście obiektu jest w zasadzie identyczne, także nie trzeba zwiększać horyzontu dynamiki. Sprawdzę jeszcze mniejszy horyzont $D = 70$:
\begin{center}
    \includegraphics[width=0.96\textwidth]{STP_projekt2_zad5_a_Dn70.png}
\end{center}
\noindent
Na powyższym wykresie można zauważyć, że w okolicy $k \approx 90$ występują pewne niezamierzone zakłócenia/regulacji. Stąd, można wysunąć wniosek, że $D = 70$ jest zbyt niskim horyzontem dynamiki. \\

\noindent
Ostatecznie, dowodzi to, że horyzont $D=80$ jest odpowiednim horyzontem dynamiki dla tego modelu.


\section*{b)}
W tym podpunkcie skrócę horyzont predykcji do minimalnej wielkości, która dalej pozwala na efektywną regulację (przy założeniu, że $N_u = N$). Zaczynam więc od $N=70$ - wykres został przedstawiony na następnej stronie.

\begin{center}
    \includegraphics[width=0.96\textwidth]{STP_projekt2_zad5_b_N70.png}
\end{center}

\noindent
Jakość regulacji jest taka sama, zmniejszam zatem wartość horyzontu predykcji:
\begin{center}
    \includegraphics[width=0.96\textwidth]{STP_projekt2_zad5_b_N60.png}
\end{center}

\clearpage \noindent
Jakość dalej jest dobra, zmniejszam zatem dalej:
\begin{center}
    \includegraphics[width=0.96\textwidth]{STP_projekt2_zad5_b_N50.png}
\end{center}

\noindent
Podobna sytuacja, zmniejszam dalej:
\begin{center}
    \includegraphics[width=0.96\textwidth]{STP_projekt2_zad5_b_N40.png}
\end{center}

\clearpage \noindent
Regulacja dalej jest bardzo dobra, zmniejszam zatem horyzont o większą wartość:
\begin{center}
    \includegraphics[width=0.96\textwidth]{STP_projekt2_zad5_b_N20.png}
\end{center}

\noindent
Zmniejszam dalej:
\begin{center}
    \includegraphics[width=0.96\textwidth]{STP_projekt2_zad5_b_N15.png}
\end{center}

\clearpage \noindent
Dla $N=15$ można zauważyć już wyraźny spade jakości regulacji. Sprawdzam zatem większe horyzonty:
\begin{center}
    \includegraphics[width=0.96\textwidth]{STP_projekt2_zad5_b_N16.png}
\end{center}
\begin{center}
    \includegraphics[width=0.96\textwidth]{STP_projekt2_zad5_b_N17.png}
\end{center}
\begin{center}
    \includegraphics[width=0.96\textwidth]{STP_projekt2_zad5_b_N18.png}
\end{center}

\noindent
Dla $N = 18$ regulacja jest już w zasadzie taka sama jak dla $N=17$. Z drugiej strony między $N=17$, a $N=16$ można zauważyć minimalną różnicę na korzyść horyzontu predykcji $N=17$. W związku z tym tę wartość horyzontu predykcji uznaję za ostateczną. \\

\noindent
Póki co, zebrano takie wartości parametrów:
\[D = 80\]
\[N = 17\]


\section*{c)}
Teraz sprawdzę wpływ horyzontu sterowania $N_u$ na jakość regulacji. Następnie wybiorę najlepszą wartość, możliwie jak najmniejszą (mniejszy horyzont sterowania oznacza mniejszą liczbę obliczeń). \\

\noindent
Kolejne wykresy zostały przedstawione na kolejnych stronach.

\clearpage \noindent
Zacznę od horyzontu $N_u = 16$, czyli o 1 mniejszy od $N$:
\begin{center}
    \includegraphics[width=0.96\textwidth]{STP_projekt2_zad5_c_Nu16.png}
\end{center}

\noindent
Jakość regulacji jest taka sama jak dla $N_u = 17$, sprawdzam więc mniejsze horyzonty: $N_u = 15, 14, 13$ :
\begin{center}
    \includegraphics[width=0.96\textwidth]{STP_projekt2_zad5_c_Nu15.png}
\end{center}

\begin{center}
    \includegraphics[width=0.96\textwidth]{STP_projekt2_zad5_c_Nu14.png}
\end{center}

\begin{center}
    \includegraphics[width=0.96\textwidth]{STP_projekt2_zad5_c_Nu13.png}
\end{center}

\clearpage \noindent
Jak można na powyższych wykresach zobaczyć, jakość regulacji jest cały czas taka sama. Dlatego sprawdzę horyzont sterowania od drugiej strony, zaczynając od $N_u = 1$:
\begin{center}
    \includegraphics[width=0.9\textwidth]{STP_projekt2_zad5_c_Nu1.png}
\end{center}

\noindent
Dla $N_u = 1$ regulator działa w sposób niestabilny. Sprawdzam zatem kolejne stopnie $N_u = 2, 3, 4$: 
\begin{center}
    \includegraphics[width=0.9\textwidth]{STP_projekt2_zad5_c_Nu2.png}
\end{center}

\begin{center}
    \includegraphics[width=0.9\textwidth]{STP_projekt2_zad5_c_Nu3.png}
\end{center}

\begin{center}
    \includegraphics[width=0.9\textwidth]{STP_projekt2_zad5_c_Nu4.png}
\end{center}

\noindent
Jak można zauważyć, dla $N_u = 2, 3, 4$ regulacja działa w zasadzie tak samo dobrze, na wykresie nie widać znaczących różnic. W związku z tym za najlepszy horyzont sterowania można uznać $N_u = 2$. Działa on tak samo jak wyższe stopnie, a wymaga najmniejszej mocy obliczeniowej. \\

\noindent
Stąd, zebrane póki co parametry regulatora przedstawiają sie jak poniżej:
\[D = 80\]
\[N = 17\]
\[N_u = 2\]


\section*{d)}
Na sam koniec sprawdzę wpływ współczynnika $\lambda$ na jakość regulacji. Wcześniej przedstawione wykresy obrazowały działanie regulatora dla $\lambda = 1$. Dla późniejszego porównania, przedstawiam trajektorię wyjścia obiektu dla $\lambda = 1$:
\begin{center}
    \includegraphics[width=0.9\textwidth]{STP_projekt2_zad5_d_lambda1.png}
\end{center}
Jak widać, zadane wyjście jest osiągane bardzo szybko, w około 30-40 chwil dyskretnych $k$. Problemem może być jednak występujące przeregulowanie. Dodatkowo, sygnał sterujący jest dosyć "poszarpany" oraz jego wartość na początku bardzo szybko wzrasta (co może być nierealizowalne fizycznie). \\

\noindent
Sprawdzę zatem inne wartości parametry $\lambda$, aby sprawdzić jaka wartość będzie odpowiednia dla różnych wymagań.

\clearpage \noindent
Sprawdzę na początku wartości niewiele wyższe od 1, dokładniej $\lambda = 2,3,4,5$:
\begin{center}
    \includegraphics[width=0.9\textwidth]{STP_projekt2_zad5_d_lambda2.png}
\end{center}
\begin{center}
    \includegraphics[width=0.9\textwidth]{STP_projekt2_zad5_d_lambda3.png}
\end{center}
\begin{center}
    \includegraphics[width=0.9\textwidth]{STP_projekt2_zad5_d_lambda4.png}
\end{center}

\begin{center}
    \includegraphics[width=0.9\textwidth]{STP_projekt2_zad5_d_lambda5.png}
\end{center}

\noindent
Wykresy zmieniają się względnie wolno, zatem teraz przedstawię większą ilość wykresów dla wyższych wartości współczynnika $\lambda$. \\

\noindent
\includegraphics[width=0.5\textwidth]{STP_projekt2_zad5_d_lambda10.png}
\includegraphics[width=0.5\textwidth]{STP_projekt2_zad5_d_lambda20.png} \\
\includegraphics[width=0.5\textwidth]{STP_projekt2_zad5_d_lambda30.png} 
\includegraphics[width=0.5\textwidth]{STP_projekt2_zad5_d_lambda40.png} \\
\includegraphics[width=0.5\textwidth]{STP_projekt2_zad5_d_lambda50.png} 
\includegraphics[width=0.5\textwidth]{STP_projekt2_zad5_d_lambda60.png} \\
\includegraphics[width=0.5\textwidth]{STP_projekt2_zad5_d_lambda70.png} 
\includegraphics[width=0.5\textwidth]{STP_projekt2_zad5_d_lambda80.png} \\
\includegraphics[width=0.5\textwidth]{STP_projekt2_zad5_d_lambda120.png} 
\includegraphics[width=0.5\textwidth]{STP_projekt2_zad5_d_lambda160.png} \\
\includegraphics[width=0.5\textwidth]{STP_projekt2_zad5_d_lambda200.png} 
\includegraphics[width=0.5\textwidth]{STP_projekt2_zad5_d_lambda240.png} \\

\noindent
Na podstawie powyższych wykresów można określić główne zmiany w działaniu regulatora wraz ze wzrostem wartości współczynnika $\lambda$:
\begin{itemize}
  \item "Przesunięcie" wykresu w prawą stroną. Moment osiągnięcia zadanego wyjścia następuje później, co oznacza, że regulacja jest wolniejsza.
  \item Wygładzenie trajektorii wyjścia oraz sygnału sterującego. Zmiany są łagodniejsze, mniej gwałtowne. W przypadku $\lambda \in <1, 20>$ na początku widać gwałtowny skok sygnału sterującego, dla większych wartości trajektoria się wygładza.
  \item Na początku następuje wzrost przeregulowania, ale przy większych wartościach $\lambda$ sytuacja się odwraca. Nawet do całkowitego wyeliminowania przeregulowania dla $\lambda = 240$.
\end{itemize}

\noindent
Podsumowując, wybranie parametru $\lambda$ będzie zależeć od warunków w jakich mamy przeprowadzać regulację. Jeżeli przeregulowanie jest dozwolone, a szybki wzrost wartości sygnału sterującego nie jest problemem, to najlepszym wyborem będzie $\lambda = 2$ (wiem, że dla $\lambda = 1$ przeregulowanie jest mniejsze, ale wybrałem $\lambda = 2$, ponieważ nie występują wtedy niepokojące szarpania sygnałem sterującym widoczne na wykresie). \\

\noindent
Jeżeli jednak nie możemy sobie pozwolić na przeregulowanie oraz gwałtowne zmiany sygnału sterującego, należy wybrać $\lambda = 240$. Czas osiągnięcia wartości zadanej jest dłuższy, ale przejście między wartościami jest bardzo płynne.


\section*{Zadanie 6 - porównanie algorytmów}
Na sam koniec porównam jakość regulacji obu algorytmów oraz wyznaczę ich obszary stabilności. \\

\noindent
Dzałanie algorytmów porównałem używając dostrojonych dyskretnych wersji algorytmów. Dla algorytmu PID parametry są równe:
\begin{align*}
    K_r & = 0,3402 \\
    T_i & = 10 \\
    T_d & = 2,4 
\end{align*}
\noindent
a dla algorytmu DMC:
\begin{align*}
    D & = 80 \\
    N & = 17 \\
    N_u & = 2 \\
    \lambda & = 2
\end{align*}

\noindent
Porównanie działanie algorytmów przedstawia się jak poniżej: \\ \\
\includegraphics[width=\textwidth]{STP_projekt2_zad6_porownanie.png}

\clearpage \noindent
Ważne dla porównania będą także przebiegi sygnału sterującego:\\
\includegraphics[width=\textwidth]{STP_projekt2_zad6_porownanie_u.png}

\noindent
Na podstawie powyższych przebiegów można sformułować poniższe wnioski:
\begin{itemize}
    \item W początkowej fazie regulacji przebiegi wyjścia $y$ wyglądają bardzo podobnie. Tak samo jest z widocznym przeregulowaniem. Różnica pojawia się dopiero później, dla algorytmu DMC wyjście prawie od razu stabilizuje się na zadanej wartości. W przypadku algorytmu PID występują jeszcze przez jakiś czas oscylacje, by osiągnąć stabilność dopiero po dłuższej chwili.
    \item Sygnał sterujący w przypadku algorytmu PID bardzo gwałtowanie "skacze" na samym początku regulacji. Dla algorytmu DMC przebieg sygnału sterującego jest dużo bardziej łagodny, różnice między kolejnymi wartościami sygnału są względnie małe.
    \item Ogólny czas regulacji jest dla algorytmu DMC jest dużo krótszy - o wiele szybciej wyjście obiektu stabilizuje się na wartości zadanej.
\end{itemize}

\noindent
Stąd, ogólnie można wysunąć wniosek, że algorytm DMC jest zdecydowanie lepszy w regulacji rozważanego obiektu. Regulacja jest dużo szybsza, a sygnał sterujący przyjmuje łatwiejszą do zrealizowania fizycznego postać.

\clearpage

\section*{Zadanie 6 - wyznaczenie obszarów stabilności}
Wyznaczę teraz obszary stabilności obu algorytmów tzn. krzywą $K_0 / K_0^{nom}$ w funkcji $T_0 / T_0^{nom}$. Jako $T_0 / T_0^{nom}$ przyjmę:
\[
    T_0 / T_0^{nom} = \{ 1; 1,1; 1,2; 1,3; 1,4; 1,5; 1,6; 1,7; 1,8; 1,9; 2,0 \}  
\]
Wyznaczę to w następujący sposób. Zastosuję wyznaczone wcześniej nastawy obu regulatorów, dla PID:
\begin{align*}
    K_r & = 0,3402 \\
    T_i & = 10 \\
    T_d & = 2,4 
\end{align*}
\noindent
a dla DMC:
\begin{align*}
    D & = 80 \\
    N & = 17 \\
    N_u & = 2 \\
    \lambda & = 2
\end{align*}

\noindent
Przyjmuję wartości $K_0^{nom}$ oraz $T_0^{nom}$ na te podane w treści zadania:
\[ K_0^{nom} = 3,4\]
\[ T_0^{nom} = 5 \]
Następnie, dla podanych stosunków $T_0 / T_0^{nom}$ będę zwiększał kolejno wartości $T_0$. Przy każdej symulacji będę cały czas korzystał z tych samych wartości współczynników. Dla każdej z wartości $T_0$ znajdę takie $K_0$ dla którego model będzie się zachowywał niestabilnie. Oznacza to osiągnięcie przebiegu niegasnących oscylacji ze stałą amplitudą (podobnie jak w metodzie Zieglera-Nicholsa). Poniżej przedstawię przykład dla pierwszej wartości $T_0$ dla obu algorytmów. \\

\noindent
Przyjęcie stałych nastawów oznacza dla algorytmu PID użycie tych samych współczynników, a dla DMC użycie macierzy $M$, $M^P$ oraz wektora $K$ wyznaczonych dla zadanego ogólnie obiektu - nie obliczam od nowa tych macierzy. \\

\noindent
Zaczynam od algorytmu PID, dla wartości stosunku $T_0 / T_0^{nom} = 1.0$. Przebiegi przedstawiono na kolejnych stronach

\clearpage \noindent
Na początku $K_0 / K_0^{nom} = 1,1$:
\begin{center}
    \includegraphics[width=0.9\textwidth]{STP_projekt2_zad6_pid_1.100.png}
\end{center}

\noindent
Regulacja jest cały czas stabilna dlatego sprawdzam kilka następnych przebiegów: \\
\includegraphics[width=0.5\textwidth]{STP_projekt2_zad6_pid_1.200.png}
\includegraphics[width=0.5\textwidth]{STP_projekt2_zad6_pid_1.300.png} \\
\includegraphics[width=0.5\textwidth]{STP_projekt2_zad6_pid_1.400.png}
\includegraphics[width=0.5\textwidth]{STP_projekt2_zad6_pid_1.500.png} \\

\noindent
Z każdym wzrostem stosunku $K_0 / K_0^{nom}$ regulacja jest coraz mniej stabilna. Mimo wszystko, dalej zadana wartość jest osiągana, sprawdzam więc zdecydowanie wyższe wartości $K_0 / K_0^{nom} = 1,8$ oraz $K_0 / K_0^{nom} = 1,9$:
\begin{center}
    \includegraphics[width=0.9\textwidth]{STP_projekt2_zad6_pid_1.800.png}
\end{center}
\begin{center}
    \includegraphics[width=0.9\textwidth]{STP_projekt2_zad6_pid_1.900.png}
\end{center}

\clearpage \noindent
Dla $K_0 / K_0^{nom} = 1,9$ regulacja jest niestabilna, sprawdzam zatem pośrednie wartości między $1,8$ a $1,9$, na początku $K_0 / K_0^{nom} = 1,85$:
\begin{center}
    \includegraphics[width=0.9\textwidth]{STP_projekt2_zad6_pid_1.850.png}
\end{center}
Regulacja jest stabilna więc sprawdzam wyższą wartość $K_0 / K_0^{nom} = 1,87$:
\begin{center}
    \includegraphics[width=0.9\textwidth]{STP_projekt2_zad6_pid_1.870.png}
\end{center}

\clearpage\noindent
Regulacja jest dalej stabilna, zwiększam zatem do $K_0 / K_0^{nom} = 1,89$:
\begin{center}
    \includegraphics[width=0.9\textwidth]{STP_projekt2_zad6_pid_1.890.png}
\end{center}

\noindent
Podobna sytuacja, sprawdzam więc $K_0 / K_0^{nom} = 1,895$:
\begin{center}
    \includegraphics[width=0.9\textwidth]{STP_projekt2_zad6_pid_1.895.png}
\end{center}

\noindent
Otrzymałem niegasnące oscylacje o stałej amplitudzie, zatem to będzie docelowa wartość. Analogiczne działania przeprowadziłem dla kolejnych wartości $T_0 / T_0^{nom}$.

\clearpage \noindent
Teraz pokażę przykładowe wyznaczanie parametru dla algorytmu DMC. Sposób wyznaczania krytycznego wzmocnienia jest podobny Dla pierwszych kilku wartości $K_0 / K_0^{nom}$ dla $T_0 / T_0^{nom} = 1,0$ przebiegi wyglądają następująco: \\
\includegraphics[width=0.5\textwidth]{STP_projekt2_zad6_dmc_1.100.png}
\includegraphics[width=0.5\textwidth]{STP_projekt2_zad6_dmc_1.200.png} \\
\includegraphics[width=0.5\textwidth]{STP_projekt2_zad6_dmc_1.300.png}
\includegraphics[width=0.5\textwidth]{STP_projekt2_zad6_dmc_1.400.png} \\

\noindent
Jak można zauważyć, wyglądają one podobnie, wzrasta tylko początkowe przeregulowanie. Sprawdzam zatem kolejne wartości $K_0 / K_0^{nom}$.:
\begin{center}
    \includegraphics[width=0.8\textwidth]{STP_projekt2_zad6_dmc_1.500.png}
\end{center}

\noindent
Otrzymywanie przebiegi są bardzo podobne, więc zwiększam wzmocnienie dużo bardziej:
\begin{center}
    \includegraphics[width=0.95\textwidth]{STP_projekt2_zad6_dmc_1.800.png}
\end{center}
\begin{center}
    \includegraphics[width=0.95\textwidth]{STP_projekt2_zad6_dmc_2.100.png}
\end{center}

\clearpage \noindent
Dla $K_0 / K_0^{nom} = 2,1$ regulacja jest już niestabilna. Sprawdzam więc mniejsze wartości:
\begin{center}
    \includegraphics[width=0.95\textwidth]{STP_projekt2_zad6_dmc_2.000.png}
\end{center}

\noindent
Oscylacje są gasnące, więc zwiększam:
\begin{center}
    \includegraphics[width=0.95\textwidth]{STP_projekt2_zad6_dmc_2.050.png}
\end{center}

\noindent
Postępując analogicznie dla kilku dalszych przebiegów mam:
\begin{center}
    \includegraphics[width=0.95\textwidth]{STP_projekt2_zad6_dmc_2.020.png}
\end{center}
\begin{center}
    \includegraphics[width=0.95\textwidth]{STP_projekt2_zad6_dmc_2.010.png}
\end{center}
\begin{center}
    \includegraphics[width=0.95\textwidth]{STP_projekt2_zad6_dmc_2.012.png}
\end{center}

\noindent
Ostatecznie dla $K_0 / K_0^{nom} = 2,012$ otrzymałem przebieg z oscylacjami o stałej amplitudzie. Uznaję więc tę wartość jako ostateczną. Analogiczne rozumowanie przeprowadziłem dla kolejnych wartości $T_0 / T_0^{nom}$. Pozwalają one na wyznaczenie obszaru stabilności algorytmu DMC. \\

\noindent
Z ciekawszych zjawisk zaobserwowanych przy wyznaczaniu wzmocnień krytycznych dla algorytmu DMC można wymienić na pewno zmianę wyglądu przebiegu dla $T_0 / T_0^{nom} \geq 1,5$. Staje się on wtedy dużo bardziej "rwany", mniej regularny. Przez to bardziej niestabilny (przebieg na następnej stronie). \\

\noindent
Dodatkowo, dla $T_0 / T_0^{nom} \geq 1,6$ występuje na początku duże przeregulowanie, aby następnie przebieg się zbliżył blisko zadanej wartości. Dopiero po zbliżeniu się do wartości zadanej występują niegasnące oscylacje o stałej amplitudzie. Jest to inne zachowanie niż to zaobserwowane wcześniej gdzie układ od razu wchodzi w takie oscylacje (efekt widać na drugim rysunku na kolejnej stronie).

\begin{center}
    \includegraphics[width=\textwidth]{STP_projekt2_zad6_dmc_1.637.png}
\end{center}
\begin{center}
    \includegraphics[width=\textwidth]{STP_projekt2_zad6_dmc_1.257.png}
\end{center}


\clearpage \noindent
Podsumowując, zebrano poniższe dane dla obu algorytmów:
\begin{table}[h!]
    \centering
    \begin{tabular}{| c || c | c |}
        \hline
        $T_0 / T_0^{nom}$ & $K_0 / K_0^{nom}$ - PID & $K_0 / K_0^{nom}$ - DMC  \\
        \hline 
        1,0 & 1,895 & 2,012 \\
        \hline
        1,1 & 1,810 & 1,973 \\
        \hline
        1,2 & 1,733 & 1,907 \\
        \hline
        1,3 & 1,663 & 1,821 \\
        \hline
        1,4 & 1,599 & 1,728 \\
        \hline
        1,5 & 1,540 & 1,637 \\
        \hline
        1,6 & 1,483 & 1,257 \\
        \hline
        1,7 & 1,430 & 1,014 \\
        \hline
        1,8 & 1,380 & 0,867 \\
        \hline
        1,9 & 1,332 & 0,791 \\
        \hline
        2,0 & 1,286 & 0,769 \\
        \hline
    \end{tabular}
\end{table}

\noindent
Porównanie tych danych na wykresie wygląda jak poniżej:
\begin{center}
    \includegraphics[width=\textwidth]{STP_projekt2_zad6_wykresy.png}
\end{center}

\noindent
Jak można zauważyć na powyższym wykresie, dla początkowych wartości $T_0 / T_0^{nom}$ bliskich 1, algorytm DMC wykazuje większą tolerancję dla zadanych obiektów o większym wzmocnieniu. Sytuacj jednak się zmienia dla $T_0 / T_0^{nom} \approx 1,5$, następuje wtedy gwałtowny spadek stosunku $K_0 / K_0^{nom}$, a algorytm DMC staje się mniej stabilny od PID. Dodatkowo, dla $T_0 / T_0^{nom} > 1,7$ widać, że algorytm DMC przestaje sobie radzić nawet dla nominalnego $K_0$. Ostatecznie dla algorytmu DMC $K_0 / K_0^{nom}$ stabilizuje się na poziomie nieco poniżej $K_0 / K_0^{nom} = 0,8$. \\

\noindent
W przypadku regulatora PID wartość $K_0 / K_0^{nom}$ maleje względnie liniowo i stabilnie. Jego stabilność jest bardziej przewidywalna niż algorytmu DMC. \\

\noindent
Podsumowując, algorytm DMC osiąga lepsze rezultaty niż algorytm PID, pod warunkiem odpowiedniego nastrojenia. W przedstawionych sytuacji jest on jednak mniej stabilny kiedy charakterystyka obiektu się zmienia względem tej której miało miejsce nastrojenie regulatora. Algorytm PID mimo nieco gorszej regulacji dla dobrze dobranych nastawów, jest bardziej stabilny w sytuacji zmiennej chrakterystyki obiektu. Jego zachowanie jest także bardziej przewidywalne, co widać m.in. na wykresie przedstawiającym obszary stabilności.


\begin{flushright}
Dziękuję za przeczytanie \\
Piotr Kostrzeński
\end{flushright}


\end{document}

