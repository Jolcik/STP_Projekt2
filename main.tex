\documentclass{article}
\usepackage[utf8]{inputenc}

\usepackage{polski}
\usepackage{amsmath}
\usepackage{graphicx}
\usepackage{mathtools}

\title{STP projekt nr 2}
\author{Piotr Kostrzeński}
\date{nr indeksu 300421}

\begin{document}

\maketitle

\section*{Wstęp}
Zadany obiekt regulacji jest opisany transmitancją:
\[
    G(s) = \frac{
        K_0 e^{-T_0s}
    }{
        (T_1s+1)(T_2s+1)
    }
\]
gdzie $K_0 = 3,4, T_0 = 5, T_1 = 1,75, T_2 = 5,51$/

\section*{Zadanie 1}
W ramach tego zadania wyznaczę transmitancję dyskretną $G(z)$ na podstawie zadanej transmitancji ciągłej $G(s)$. Zastosuję przy tym ekstrapolator zerowego rzędu oraz okres próbkowania $T_p = 0,5s$. Zadaną transmitancję ciągłą można także przedstawić jako:
\[
    G(s) = e^{-T_0s} \cdot G_0(s)
\]
gdzie:
\[
    G_0(s) = \frac{K_0}{(T_1s+1)(T_2s+1)}
\]
Transmitancję $G_0(s)$ rozkładam na ułamki proste:
\begin{align*}
    G_0(s) & = \frac{K_0}{(T_1s+1)(T_2s+1)} = \frac{\frac{K_0}{T_1T_2}}{(s+\frac{1}{T_1})(s+\frac{1}{T_2})} =  \frac{A}{s+\frac{1}{T_1}} + \frac{B}{s+\frac{1}{T_2}} = \\
    & = \frac{A(s+\frac{1}{T_2}) + B(s+\frac{1}{T_1})}{(s+\frac{1}{T_1})(s+\frac{1}{T_2})} = \frac{(A+B)s + \frac{A}{T_2} + \frac{B}{T_1}}{(s+\frac{1}{T_1})(s+\frac{1}{T_2})}
\end{align*}

\noindent
Przyrównując współczynniki wielomianu w liczniku ułamka otrzymuję układ równań:
\[
    \begin{cases}
        A + B = 0 \\
        \frac{A}{T_2} + \frac{B}{T_1} = \frac{K_0}{T_1T_2}
    \end{cases}
\]
\[
    \begin{cases}
        A + B = 0 \\
        \frac{AT_1 + BT_2}{T_1T_2} = \frac{K_0}{T_1T_2}
    \end{cases}
\]
\[
    \begin{cases}
        A + B = 0 \implies B = - A \\
        AT_1 + BT_2 = K_0
    \end{cases}
\]
\[ AT_1 - AT_2 = K_0\]
\[ (T_1 - T_2)A = K_0\]
\[ A = \frac{K_0}{T_1 - T_2} \]
\[ B = -A = -\frac{K_0}{T_1 - T_2}\]
Podsumowując współczynniki:
\[
    \begin{cases}
         A = \frac{K_0}{T_1 - T_2} \\
         B = -\frac{K_0}{T_1 - T_2}
    \end{cases}
\]
co mi daje:
\[
    G_0(s) = \frac{\frac{K_0}{T_1 - T_2}}{s+\frac{1}{T_1}} + \frac{-\frac{K_0}{T_1 - T_2}}{s+\frac{1}{T_2}}
\]
Transmitancję ciągłą zamieniam na dyskretną według wzoru:
\[
    G_0(z) = \frac{z-1}{z} \cdot Z \left( \frac{G_0(s)}{s} \right)
\]
Daje to:

\begin{align*}
    G_0(z) & = \frac{z-1}{z} \cdot Z \left( \frac{\frac{K_0}{T_1 - T_2}}{s(s+\frac{1}{T_1})} + \frac{-\frac{K_0}{T_1 - T_2}}{s(s+\frac{1}{T_2})} \right) = \\
    & = \frac{z-1}{z} \cdot Z \left( \frac{T_1K_0}{T_1 - T_2}\frac{\frac{1}{T_1}}{s(s+\frac{1}{T_1})} - \frac{T_2K_0}{T_1 - T_2}\frac{\frac{1}{T_2}}{s(s+\frac{1}{T_2})} \right) = \\
    & = \frac{z-1}{z} \left[ 
        \frac{T_1K_0}{T_1 - T_2}\frac{z(1-e^{-\frac{1}{2T_1}})}{(z-1)(z-e^{-\frac{1}{2T_1}})} -
        \frac{T_2K_0}{T_1 - T_2}\frac{z(1-e^{-\frac{1}{2T_2}})}{(z-1)(z-e^{-\frac{1}{2T_2}})}
    \right] \\
    & = \frac{T_1K_0}{T_1 - T_2} \cdot \frac{1-e^{-\frac{1}{2T_1}}}{(z-e^{-\frac{1}{2T_1}})} - \frac{T_2K_0}{T_1 - T_2} \cdot \frac{1-e^{-\frac{1}{2T_2}}}{(z-e^{-\frac{1}{2T_2}})} \\
    & = \frac{K_0}{T_1 - T_2} \left( \frac{T_1(1-e^{-\frac{1}{2T_1}})}{(z-e^{-\frac{1}{2T_1}})} - \frac{T_2(1-e^{-\frac{1}{2T_2}})}{(z-e^{-\frac{1}{2T_2}})} \right) \\ 
    & = \frac{K_0}{T_1 - T_2} \cdot
    \frac{
        T_1(1-e^{-\frac{1}{2T_1}})(z-e^{-\frac{1}{2T_2}}) -
        T_2(1-e^{-\frac{1}{2T_2}})(z-e^{-\frac{1}{2T_1}})
    }{
        (z-e^{-\frac{1}{2T_1}})(z-e^{-\frac{1}{2T_2}})
    } \\
    & = \frac{K_0}{T_1 - T_2} \cdot
    \frac{
        \left[T_1(1-e^{-\frac{1}{2T_1}})-T_2(1-e^{-\frac{1}{2T_2}})\right]z -
        e^{-\frac{1}{2T_1}}T_2(1-e^{-\frac{1}{2T_2}}) - e^{-\frac{1}{2T_2}}T_1(1-e^{-\frac{1}{2T_1}})
    }{
        (z-e^{-\frac{1}{2T_1}})(z-e^{-\frac{1}{2T_2}})
    } 
\end{align*}
\noindent
Ostatecznie:
\begin{align*}
    G_0(z) = \frac{K_0}{T_1 - T_2} \cdot
    \frac{
        \left[T_1(1-e^{-\frac{1}{2T_1}})-T_2(1-e^{-\frac{1}{2T_2}})\right]z -
        e^{-\frac{1}{2T_1}}T_2(1-e^{-\frac{1}{2T_2}}) - e^{-\frac{1}{2T_2}}T_1(1-e^{-\frac{1}{2T_1}})
    }{
        z^2 - (e^{-\frac{1}{2T_1}} + e^{-\frac{1}{2T_2}})z + e^{-\frac{1}{2T_1}}e^{-\frac{1}{2T_2}}
    }
\end{align*}

\noindent
Po podstawieniu liczb otrzymuję taką transmitancję dyskretną:
\begin{align*}
    G_0(z) & = -0,90425 \cdot \frac{
       (0,43491 - 0,47798)z + 0,35919 - 0,39719 
    }{
       z^2 - 1,66473z + 0,68628
    } \\
    & = -0,90425 \cdot \frac{
       -0,04307z - 0,038
    }{
       z^2 - 1,66473z + 0,68628
    } \\ 
    & = \frac{
       0,03894z + 0,0343 
    }{
       z^2 - 1,66473z + 0,68628
    }
\end{align*}
Otrzymaną transmitancję sprawdziłem także w środowisku MATLAB. Najpierw przekształciłem tranmitancję ciągłą:
\begin{align*}
    G_0(s) & = \frac{\frac{K_0}{T_1T_2}}{(s+\frac{1}{T_1})(s+\frac{1}{T_2})} \\
    & = \frac{\frac{K_0}{T_1T_2}}{s^2+(\frac{1}{T_1} + \frac{1}{T_2})s + \frac{1}{T_1T_2}}
\end{align*}
Po podstawieniu liczb:
\[
G_0(s) = \frac{0,3526}{s^2+0,7529s + 0,1037}
\]
Transmitancję w takiej postaci można w środowisku MATLAB zamienić na transmitancję dyskretną w poniższy sposób:
\begin{verbatim}
    H = tf([0.3526], [1 0.7529 0.1037]);
    Hd = c2d(H, 0.5);
\end{verbatim}
Wynik powyższych obliczeń wygląda nastepująco:
\begin{verbatim}
    Hd =
 
       0.03895 z + 0.03435
      ----------------------
      z^2 - 1.665 z + 0.6863
\end{verbatim}
\noindent
Co potwierdza poprawność wyżej wykonanych obliczeń. Pozostało jeszcze zająć się pominiętym póki co opóźnieniem. Po przyjęciu okresu próbkowania $T_p = 0,5s$ opóźnieniu ciągłemu $e^{-5s}$ odpowiada opóźnienie dyskretne $z^{-10}$. Stąd ostatecznie, transmitancja dyskretna $G(z)$ przyjmuje postać:
\[
    G(z) = \frac{
       0,03894z + 0,0343 
    }{
       z^2 - 1,66473z + 0,68628
    } z^{-10}
\]

\clearpage \noindent
Obie transmitancje zostały zaimplementowane w środowisku MATLAB. Nastepnie sprawdzono ich odpowiedzi skokowe. Wyglądają one następująco: \\
\includegraphics[width=\textwidth]{STP_projekt2_zad1.png}

\noindent
Jak można zauważyć, transmitancję się praktycznie ze sobą pokrywają (oczywiście biorąc pod uwagę to że jedna jest dyskretna). Dowodzi to poprawności przeprowadzonej dyskretyzacji transmitancji. Współczynnik wzmocnienia statycznego transmitancji można policzyć z poniższych wzorów, dla transmitancji ciągłej i dyskretnej:
\[
    K_{stat} = \lim_{s \to 0} G(s), K_{stat} = \lim_{z \to 1} G(z)
\]
Stąd, dla transmitancji ciągłej:
\[
    K_{stat} = \lim_{s \to 0}  \frac{0,3526}{s^2+0,7529s + 0,1037} e^{-5s} = \frac{0,3526}{0,1037} \approx 3,4
\]
i dla dyskretnej:
\begin{align*}
    K_{stat} & = \lim_{z \to 1} \frac{
       0,03894z + 0,0343 
    }{
       z^2 - 1,66473z + 0,68628
    } z^{-10} \\
    & = \frac{
       0,03894 + 0,0343 
    }{
       1^2 - 1,66473\cdot1 + 0,68628
    } 1^{-10} = \frac{0,07324}{0.02155} \approx 3,3986 \approx 3,4
\end{align*}

\noindent
Obliczone odpowiedzi skokowe są prawie identyczne (błędy związane z zaokrągleniem). Zatem wartości współczynnika wzmocnienia statycznego transmitancji ciągłej i dyskretnej są sobie równe. Można to także zauważyć, na wykresie z odpowiedziami skokowymi. Widać na nim, że odpowiedź skokowa dla obu transmitancji stabilizuje się po pewnym czasie na obliczonych wartościach.

\section*{Zadanie 2}

    

\end{document}

